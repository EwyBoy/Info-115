\documentclass[11pt]{article}

\usepackage[utf8]{inputenc}
\usepackage{hyperref}
\usepackage{graphicx}

\title{ Compulsory Assignment 1}
\author{ Petter Jakub Økland - 223073\\
Eivind Norling - 267792}

\begin{document}
\maketitle

\section{The Karate Club Network}
\subsection{}
\begin{figure}
  \includegraphics[width=\linewidth]{Figure_1.png}
  \caption{A graph.}
  \label{fig:graph model}
\end{figure}

Figure \ref{fig:graph model} shows a graph.

\subsection{}
Zachary\cite{Zachary} analyzes the splintering of a karate club into two different organizations
over a dispute over lessen fees. He lists several reasons for formalizing this fission
process into a mathematical model, the most important perhaps being that the model is
new to anthropology.\\
\\
His analysis shows how the flow of political information predicts the two opposing
factions' increasingly divergent views and strategy ultimately leading to a fission
as information flow between factions becomes increasingly hampered by a bottleneck
of information between them.\\
\\
Beyond just club meetings, shared perceptions on the nature of the club communicated
in the friendship network was unable to become reconciled accross factional boundraries.
An organizational and ideological feedback relationship -- a vicious cycle-- wherein the
strategy of the factions increasingly intensified the ideological divide while threatening
the organizational basis for the network's existence itself made the fission all but
inevitable.\\
\\
The form of this analysis can be applied to more than just the karate club, extending to
analysis of communication patterns in small groups in general due to the inherency of the
the feature of the minimum cut in a capacitated network in this mathematical model, which
is what made this model quite new to anthropology at the time. Whenever the existence of
a unique minimum cut in a capacitated network can be established, it will under certain
conditions (here communication of club meetings) result in a barrier to group unity which
can end up in a group fission.

\section{}
Running the following functions yielded the following results:\\
\\
degree\_centrality(G)\\
1: 0.48484848484848486, 2: 0.2727272727272727, 3: 0.30303030303030304,
4: 0.18181818181818182, 5: 0.09090909090909091, 6: 0.12121212121212122,
7: 0.12121212121212122, 8: 0.12121212121212122, 9: 0.15151515151515152,
10: 0.06060606060606061, 11: 0.09090909090909091, 12: 0.030303030303030304,
13: 0.06060606060606061, 14: 0.15151515151515152, 15: 0.06060606060606061,
16: 0.06060606060606061, 17: 0.06060606060606061, 18: 0.06060606060606061,
19: 0.06060606060606061, 20: 0.09090909090909091, 21: 0.06060606060606061,
22: 0.06060606060606061, 23: 0.06060606060606061, 24: 0.15151515151515152,
25: 0.09090909090909091, 26: 0.09090909090909091, 27: 0.06060606060606061,
28: 0.12121212121212122, 29: 0.09090909090909091, 30: 0.12121212121212122,
31: 0.12121212121212122, 32: 0.18181818181818182, 33: 0.36363636363636365,
34: 0.5151515151515151
\\\\
betweenness\_centrality(G)\\
1: 0.43763528138528146, 2: 0.053936688311688304, 3: 0.14365680615680618,
4: 0.011909271284271283, 5: 0.0006313131313131313, 6: 0.02998737373737374,
7: 0.029987373737373736, 8: 0.0, 9: 0.05592682780182781, 10: 0.0008477633477633478,
11: 0.0006313131313131313, 12: 0.0, 13: 0.0, 14: 0.04586339586339586, 15: 0.0,
16: 0.0, 17: 0.0, 18: 0.0, 19: 0.0, 20: 0.03247504810004811, 21: 0.0, 22: 0.0,
23: 0.0, 24: 0.017613636363636363, 25: 0.0022095959595959595, 26: 0.0038404882154882154,
27: 0.0, 28: 0.02233345358345358, 29: 0.0017947330447330447, 30: 0.0029220779220779218,
31: 0.014411976911976909, 32: 0.13827561327561325, 33: 0.145247113997114, 34: 0.30407497594997596
\\\\
closeness\_centrality(G)\\
1: 0.5689655172413793, 2: 0.4852941176470588, 3: 0.559322033898305,
4: 0.4647887323943662, 5: 0.3793103448275862, 6: 0.38372093023255816,
7: 0.38372093023255816, 8: 0.44, 9: 0.515625, 10: 0.4342105263157895,
11: 0.3793103448275862, 12: 0.36666666666666664, 13: 0.3707865168539326,
14: 0.515625, 15: 0.3707865168539326, 16: 0.3707865168539326, 17: 0.28448275862068967,
18: 0.375, 19: 0.3707865168539326, 20: 0.5, 21: 0.3707865168539326,
22: 0.375, 23: 0.3707865168539326, 24: 0.39285714285714285, 25: 0.375,
26: 0.375, 27: 0.3626373626373626, 28: 0.4583333333333333, 29: 0.4520547945205479,
30: 0.38372093023255816, 31: 0.4583333333333333, 32: 0.5409836065573771, 33: 0.515625, 34: 0.55
\\\\
average\_clustering(G)\\
0.5706384782076823
\\\\
connectivity.local\_edge\_connectivity(G, 1, 34)\\
10\\
\\\\
As can be seen bla bla.

\section{Polarized blogging about climate change}


\section{The psychology of social media}

\subsection{}

We chose these 10 sources to articles, videos or posts that are concerned with the negative consequences and downsides of the usage social media and how it can affect our mental health. We think they cover a wide variety of different aspect, research and side effects to your mental health regarding social media. 
	
Source number 1: 
https://www.youtube.com/watch?v=tTPVNncnv0Y

The link above is a YouTube video produced by BuzzFeed about Aria Inthavong who live-streamed his entire life for 7 days, 24 hour a day onto social media. For the duration of the livestream he had a couple thousand people watching and interacting with the stream at all times, reacting to all his moves and judging him by his actions. Aria himself describes this as one of his life's most mentally challenging weeks and describes how shearing everything in your life online broke him down his mental health.

Source number 2: https://medium.com/thrive-global/how-technology-hijacks-peoples-minds-from-a-magician-and-google-s-design-ethicist-56d62ef5edf3

Source number two is a article written by Tristan Harris. He is the co-founder of Center for Humane Technology, previously worked for Google and has a broad technological background. In his article, “How Technology is Hijacking Your Mind — from a Magician and Google Design Ethicist” he talks about how technology is designed to exploit people's mind and weaknesses. “WRITE SOMETHING HERE ABOUT IN CORRELATING TO MAGIC TRICKS”

Source number 3:
https://www.thenewatlantis.com/publications/hiding-behind-the-screen

The third source I want to bring attention to is this article from the summer of 2010. The article is written by Roger Scruton who is a English writer and philosopher. In his article “Hiding Behind the Screen” he talks about human relations and how people's behavior change when they can hide behind a screen or an avatar. He also dives into the subject of how people are seeking emotional attention on the internet.

Source number 4:
https://www.youtube.com/watch?v=J54k7WrbfMg

The fourth source is a another YouTube video. It is published on the youtube channel Ewafa and the video is about an interview with Sean Parker and Chamath Palihapitiya. The video is named “Facebook is ‘Ripping Apart Society’”. The main topics in this video is about how social media, in this case Facebook is creating what Sean Parker is describing as a social validation feedback loop and how the big companies are tailoring their algorithms to exploit the psychology of the human brain on how we crave recognition and the need for “Likes” for that quick fix of a dopamine rush. Chamath talks a lot about the same but digs a bit deeper into the phycology part. He speaks about how humans not wants instant feedback and how we have lost our patience.

Source number 5: https://www.benthamsgaze.org/2016/05/12/on-the-hunt-for-facebooks-army-of-fake-likes/

The fifth source we have chosen is a more technical article about a phenomenon called “Like Farms”. The article does not dig much into the physiological aspects of farming and buying likes but we still find this article relevant to the subject and we want to look a bit deeper into this corner of the internet. “Like Farms” are websites you can go to, to buy likes on social media. These farms usually target sites like Facebook, Twitter and Instagram. The farming sites allow you to buy x number of likes for your instagram post, facebook page etc. Some farms operate with hordes of stolen, hacked or hijacked accounts controlled by bots to deliver all the likes, while other farms are employing hundred or thousands of people from India with their own personal army of fake accounts. We find it incredible that people are so addicted to these imaginary points we call likes that they are willing to go to the extent of buy them. It also amazes us that there is in fact people on this globe whos job is to create fake social media accounts and go into paying customers media posts and pages just to give em a dopamine rush and some fake acknowledgement.

Source number 6:
https://www.psychologytoday.com/us/blog/love-digitally/201603/do-you-crave-facebook-likes

The sixth source is from a article we found on Psychology Today and is written in March 2016 by Martin Graff who is a professor in digital love. In his article named “Do You Crave Facebook Likes?” he talks about his study where he looked into what aspects of validation people seeked from social media on how it affected their self-esteem and personality.

Source number 7:
https://www.psychologytoday.com/us/blog/in-excess/201805/addicted-social-media

The seventh source is an article from Psychology Today. This one is written by professor Mark D. Griffiths and was written in May 2018. This article named “Addicted to Social Media?” brings up the topic of what can be done to reduce the excessive use of social media.


Source number 8:
https://www.youtube.com/watch?v=kFSwDtspY5c

The eight source is a YouTube video that have gone viral. The video was posted August 8, 2016 on the YouTube channel “Freedom in Thought” with the title “WHY I QUIT SOCIAL MEDIA FOR A YEAR AND WHAT I LEARNED”. In his video he describe that he quit social media because he was compulsively checking his phone for social media updates. He describes how he before replaced a lot of the things he used to enjoy in life and tried to kill both time and socially awkward situations by hiding behind his phone. The first half of the video he describes how the social media machinery is wired to abuse weaknesses in our brain and constructed to consume as much attention from each user as possible to keep them on the platforms for as long as possible.
In the second half of the video he talks about how quitting social media helped him stay focused for extended amounts of time, relieved stress and created a distraction free environment so he could focus his energy into other things he enjoy.

Source number 9:
https://www.lifehack.org/604399/how-we-are-actively-making-fake-friends-on-the-social-media

The ninth source we want to bring to attention is this blog-post posted on Lifehack by Heather Poole. In her post she talks about how we build these huge networks of “fake” friends over the years while using Facebook. Everytime we meet a new person we add each other on social media platforms even though we never see or speak to that person again. We have friends we knew from middle school that we have not seen in 8 years but still like and comment on each others photos. We only see the sides of life the person want to shear with the world. On social media everyone paints their own version of themself and for most people that is the happy side where everything is glitter and gold. She suggest to filter out and delete the friends you haven’t had a offline conversation to do with in the last year. Heather claims this is a good way to keep a healthy relationship to your social media friends.

Source number 10:
https://www.economist.com/graphic-detail/2018/05/18/how-heavy-use-of-social-media-is-linked-to-mental-illness?fsrc=scn/tw/te/bl/ed/?fsrc=scn/tw/te/bl/ed/howheavyuseofsocialmediaislinkedtomentalillnessdailychart


The tenth and final source is a article from The Economist named “How heavy use of social media is linked to mental illness” and the article is talking about exactly what the title implies. The article refers a graph showcasing the different positive and negative aspects of four different social media sites. The sites are Instagram, Snapchat, Facebook and Twitter and it uses color to describe how positive or negative the different aspects are affecting people from 14 to 24 years-old. They have different factors like sleep, bullying and body image.


Summery
To summarize all the the ten sources above, they all are the result of the big corporations created the social medias trying to keep you on their site for as long as possible. Apps where you maintain fake relationships and present your fake life to fake friends.  We get addicted to these interactions and crave the need for instant recognition. Some are even willing to pay for it, for a quick fix of dopamine and bragging rights. This constant social validation feedback loops are what is fuling the social media machinergy. You stay at the big corporations applications, they earn more money from your data habits and by drowning you with ads. Then they revenu earned goes back into research on how to manipulate you even more and keep you on their platforms.
Every time you refresh a page, you are awarded with new content and information. This is food for your brain and our brains love to learn. This is triggering our dopamine receptors, and as we continue to repeat this loop we keep strengthening this loop, and addiction.
The main thing we can see from all these sources is that this is simply not good for us. If we look at Arias video from BuzzFeed we can see that concentrated amounts of social media interaction can break down a human emotialy in pretty fast. While excessive and compulsive usage of social media has many downsides, Freedom in Thought’s video shows us that completely abandoning social media can both be a positive and healing experience that can make you more happy.


\pagebreak
\bibliography{references.bib}
\bibliographystyle{ieeetr}

\end{document}
